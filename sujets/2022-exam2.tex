\documentclass[11pt,a4paper,addpoint]{exam}
\usepackage[T1]{fontenc}
\usepackage[utf8]{inputenc}
\usepackage[]{lmodern}
\usepackage[french]{babel}
%\usepackage{mathtools}
\usepackage[margin=2cm]{geometry} %layout
\usepackage{graphicx}
\usepackage{booktabs} % for much better looking tables
% Put the bibliography in the ToC
\usepackage[nottoc,notlof,notlot]{tocbibind}
% Alter the style of the Table of Contents
\usepackage[titles]{tocloft}
\usepackage{multicol}

\usepackage{minted}
\usepackage{titling}
\setlength{\droptitle}{-3cm}

\nopointsinmargin
\pointformat{}

\usepackage[pdfauthor={Axel Viala},
  pdftitle={Seconde-session 2021-2022: L2 Programmation avancée},
  pagebackref=true,%
  colorlinks=true,%
  linkcolor=green,%
  %urlcolor=green!70!black,
  pdftex]{hyperref}
\usepackage[ampersand]{easylist}
\renewcommand{\solutiontitle}{}

\author{\normalsize{Axel Viala <axel.viala@darnuria.eu>}}
\title{\normalsize{\textbf{Seconde-session 2021-2022: L2 Programmation avancée}}}

\begin{document}
  \maketitle
  \begin{multicols}{2}

  \makebox[\textwidth][l]{Nom et Prénom:}
  \makebox[\textwidth][l]{Numéro étudiant:}
  \end{multicols}
  \textbf{Objectifs:} La clarté des réponses sera appréciée, veillez à écrire soigneusement. Les questions portent sur le langage Rust. Réponse sur une copie a part encouragée. 1 pts par question. Justification encouragée.
  \begin{questions}
    \section{Généralités}

%%%%%%%%%%%%%%%%%%%%%%%%%%%%%%%%%%%%%%%%%%%%%%%%%%%%%%%%%%%%%%%%%%%%%%%%%%%%%%%%

\question[1] Typage en Rust est : (plusieurs réponses valides():
\begin{multicols}{2}
\begin{checkboxes}
    \choice Dynamique à l'exécution
    \CorrectChoice inféré à la compilation
    \choice Avec classes comme en Java
    \CorrectChoice Explicite dans les fonctions, traits, structures et enumerations
    \choice Permet de l’héritage comme en Java
    \CorrectChoice Permet des types sommes (énums)
    \choice Explicit partout comme en C
    \CorrectChoice Traçage par le compilateur des partages de références
    \choice Pas typé
\end{checkboxes}
\end{multicols}

\question[1] Une référence peut elle avoir une valeur \texttt{null} comme en C? Justifiez?
\vspace{1in}

\question[1] Dans le code ci dessous, qu'afficherait le
programme s'il appelait \mintinline{rust}{mystere(Some(1 + 1))} une fois? Quelle est la valeur de retour de mystère avec cet appel?
\begin{minted}{rust}
fn mystere(a: Option<i32>) -> i32 {
    match a {
        Some(n) => { println!("Mon mystère: {}", n); n + 42 },
        None => 0,
    }
}
\end{minted}
\vspace{1in}
%%%%%%%%%%%%%%%%%%%%%%%%%%%%%%%%%%%%%%%%%%%%%%%%%%%%%%%%%%%%%%%%%%%%%%%%%%%%%%%%

\question[1] Dans ce code, \mintinline{rust}{m} est passé comment?
\begin{checkboxes}
    \CorrectChoice Par déplacement \textit{move}
    \choice Par référence \textit{borrow} (mut/immutable)
    \choice Par copie \textit{copy}
\end{checkboxes}
\begin{minted}{rust}
struct MystereStruct { secret: i32 }
fn mystere(m: &mut MystereStruct) {
    m.secret += 1;
}
\end{minted}

%%%%%%%%%%%%%%%%%%%%%%%%%%%%%%%%%%%%%%%%%%%%%%%%%%%%%%%%%%%%%%%%%%%%%%%%%%%%%%%%
\question[1] Toute valeur par exemple une \mintinline{rust}{struct A} hors type de base est:
 \begin{checkboxes}
    \choice Pris par référence
    \CorrectChoice Déplacé
    \choice Copié
\end{checkboxes}

%%%%%%%%%%%%%%%%%%%%%%%%%%%%%%%%%%%%%%%%%%%%%%%%%%%%%%%%%%%%%%%%%%%%%%%%%%%%%%%%
\question[1] Ce code peut-il compiler si non pourquoi?
\begin{minted}{rust}
struct MystereStruct { secret: i32 }
fn add_thing(m: &mut MystereStruct, o: &MystereStruct, thing: i32) {
    m.secret += thing + o.secret;
}

fn main() {
    let mut a = MystereStruct { secret : 0 };
    add_thing(&mut a, &a, 1);
}
\end{minted}
\vspace{2in}

%%%%%%%%%%%%%%%%%%%%%%%%%%%%%%%%%%%%%%%%%%%%%%%%%%%%%%%%%%%%%%%%%%%%%%%%%%%%%%%%


%%%%%%%%%%%%%%%%%%%%%%%%%%%%%%%%%%%%%%%%%%%%%%%%%%%%%%%%%%%%%%%%%%%%%%%%%%%%%%%%
\question[1] En Rust comment exprimez l'absence de quelque chose, en C on utiliserait probablement \texttt{null}, 
donnez un exemple de code, par exemple une structure personne qui peux avoir une date de naissance ou non.
\vspace{1in}

%%%%%%%%%%%%%%%%%%%%%%%%%%%%%%%%%%%%%%%%%%%%%%%%%%%%%%%%%%%%%%%%%%%%%%%%%%%%%%%%
\question[1] Anatomie d'un code Rust : associez les termes suivants au code suivant:
\begin{multicols}{2}
\begin{itemize}
    \item Opérateur d'addition
    \item Nom de variable de match
    \item Mot clef de pattern matching/filtrage
    \item Argument de fonction
    \item Type
    \item Mot clef de déclaration de fonction
    \item Constructeurs de variant de l'enum \texttt{Option}
    \item Bloc du corps de la fonction
    \item Nom de fonction
    \item Appel de fonction associée à un type
    \item Argument de fonction passé en appel
    \item Expression du if
\end{itemize}
\end{multicols}
\begin{minted}{rust}
fn inconnue(o: Option<i32>) -> Option<i32> {
    match o {
        Some(i) if i > 18 => Some(i + 50),
        _ => None,
    }
}
\end{minted}
\vspace{2in}

%%%%%%%%%%%%%%%%%%%%%%%%%%%%%%%%%%%%%%%%%%%%%%%%%%%%%%%%%%%%%%%%%%%%%%%%%%%%%%%%
\question[1] Que signifie \mintinline{rust}{T} dans la signature de la fonction \mintinline{rust}{fn mystere<T>(a: T)}.
\vspace{2in}

%%%%%%%%%%%%%%%%%%%%%%%%%%%%%%%%%%%%%%%%%%%%%%%%%%%%%%%%%%%%%%%%%%%%%%%%%%%%%%%%
\question[1] Ce code peut-il compiler? Justifiez.
\begin{minted}{rust}
.
fn mystere(mut a: Box<[i32]>, b: &i32) {
    a[0] += *b;
}

fn main() {
    // Box<T> n'implémente pas Copy
    let a = Box::new([1, 2, 3]);
    mystere(a, &a[0]);
    println!("{:?}", a);
}
\end{minted}
\vspace{1in}
%%%%%%%%%%%%%%%%%%%%%%%%%%%%%%%%%%%%%%%%%%%%%%%%%%%%%%%%%%%%%%%%%%%%%%%%%%%%%%%%

%%%%%%%%%%%%%%%%%%%%%%%%%%%%%%%%%%%%%%%%%%%%%%%%%%%%%%%%%%%%%%%%%%%%%%%%%%%%%%%%
\question[1] Expliquez ce qu'est un \mintinline{rust}{trait} en Rust donnez un exemple.
\vspace{1in}

%%%%%%%%%%%%%%%%%%%%%%%%%%%%%%%%%%%%%%%%%%%%%%%%%%%%%%%%%%%%%%%%%%%%%%%%%%%%%%%%

\question[1] A quoi sert le type \mintinline{rust}{Option<T>} et le type \mintinline{rust}{Result<T, E>}
\vspace{1in}

%%%%%%%%%%%%%%%%%%%%%%%%%%%%%%%%%%%%%%%%%%%%%%%%%%%%%%%%%%%%%%%%%%%%%%%%%%%%%%%%
\question[1] Proposez votre implémentation de \mintinline{rust}{Result<T, E>} qui fait comme \mintinline{rust}{Tesult},
Et implémenter la fonction \mintinline{rust}{Result::map} qui dois faire comme la documentation de \mintinline{rust}{Result::map}
de la lib standard.
\vspace{3in}

%%%%%%%%%%%%%%%%%%%%%%%%%%%%%%%%%%%%%%%%%%%%%%%%%%%%%%%%%%%%%%%%%%%%%%%%%%%%%%%%
\question[1] \mintinline{rust}{bool} est il passé par copie ou par move par défaut.
\vspace{1in}

%%%%%%%%%%%%%%%%%%%%%%%%%%%%%%%%%%%%%%%%%%%%%%%%%%%%%%%%%%%%%%%%%%%%%%%%%%%%%%%%
\question[1] Ce code comporte une ou plusieurs erreurs, laquelle justifiez. Que fait \mintinline{rust}{?}. Justifiez.
\begin{minted}{rust}
fn mystere(a: Option<u32>) -> Option<u32> {
    let a = a?;
    None(a + 1)
}
\end{minted}
\vspace{1.5in}

%%%%%%%%%%%%%%%%%%%%%%%%%%%%%%%%%%%%%%%%%%%%%%%%%%%%%%%%%%%%%%%%%%%%%%%%%%%%%%%%
\question[1] Pour un point réalisez une implémentation du trait \mintinline{rust}{Debug} et \mintinline{rust}{Mul} entre deux \mintinline{rust}{Point}.
\begin{minted}{rust}
struct Triangle {
    x: f32,
    y: f32,
}

\end{minted}

%%%%%%%%%%%%%%%%%%%%%%%%%%%%%%%%%%%%%%%%%%%%%%%%%%%%%%%%%%%%%%%%%%%%%%%%%%%%%%%%
\section{Robot Quest}

\question[1] Implémentez une enum \texttt{Mouvement} qui dois représenter \texttt{Haut, Bas, Droite, Gauche, Avance}.
\question[1] Programmez une structure représentant un personnage pouvant avoir une position dans l'espace et une orientation.
\question[1] Réaliser une fonction qui si on lui passe un \texttt{\&[Mouvement]} permet de déplacer un personnage.
\question[1] Écrivez une fonction qui permet de réaliser des déplacements dans un monde en 2 dimensions, (a vous de choisir la structure) et que si le robot atteint un bord,
son mouvement n'est pas effectué il n'avance pas et ne fait pas planter le programme, et du \texttt{Monde} est libre, un mouvement par tour de boucle jusque à ce que on ai parcouru tout les mouvements.
\question[1] Ajouter dans votre représentation du monde des coffres en position (x, y): (1, 1), (5, 6). Si le robot tombe sur un coffre il dois \texttt{println!("Coffre trouvé")}
La representation des coffres est libre.
\end{questions}
\end{document}


