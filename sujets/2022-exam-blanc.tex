\documentclass[11pt,a4paper,addpoint]{exam}
\usepackage[T1]{fontenc}
\usepackage[utf8]{inputenc}
\usepackage[]{lmodern}
\usepackage[french]{babel}
%\usepackage{mathtools}
\usepackage[margin=2cm]{geometry} %layout
\usepackage{graphicx}
\usepackage{booktabs} % for much better looking tables
% Put the bibliography in the ToC
\usepackage[nottoc,notlof,notlot]{tocbibind}
% Alter the style of the Table of Contents
\usepackage[titles]{tocloft}
\usepackage{multicol}

\usepackage{minted}
\usepackage{titling}
\setlength{\droptitle}{-3cm}

\nopointsinmargin
\pointformat{}

\usepackage[pdfauthor={Axel Viala},
  pdftitle={Seconde-session 2021-2022: L2 Programmation avancée},
  pagebackref=true,%
  colorlinks=true,%
  linkcolor=green,%
  %urlcolor=green!70!black,
  pdftex]{hyperref}
\usepackage[ampersand]{easylist}
\renewcommand{\solutiontitle}{}

\author{\normalsize{Axel Viala <axel.viala@darnuria.eu>}}
\title{\normalsize{\textbf{Seconde-session blanc 2021-2022: Programmation avancée}}}

\begin{document}
  \maketitle
  \begin{multicols}{2}

  \makebox[\textwidth][l]{Nom et Prénom:}
  \makebox[\textwidth][l]{Numéro étudiant:}
  \end{multicols}
  \textbf{Objectifs:} La clarté des réponses sera appréciée, veillez à écrire soigneusement. Les questions portent sur le langage Rust.
  Notes de cours autorisé. Réponse sur une copie a part encouragée. Ce sujet blanc ne comporte pas la question de composition et les questions sont "un peu" plus dures
  que sur le sujet prévu.
  \newline
  \begin{questions}

    \section{Généralités}

%%%%%%%%%%%%%%%%%%%%%%%%%%%%%%%%%%%%%%%%%%%%%%%%%%%%%%%%%%%%%%%%%%%%%%%%%%%%%%%%

    \question[1] Par défaut doit t'on toujours écrire les types explicitement partout en Rust
    \begin{checkboxes}
        \CorrectChoice Non dans le corps des fonctions ne n'est pas neccessaire
        \choice Il faut écrire les types partout aucune inférence
    \end{checkboxes}

\question[1] En Rust une référence peut-elle pointer sur rien?:
 \begin{checkboxes}
    \choice Une référence peut pointer sur rien temporairement
    \CorrectChoice Les références doivent toujours référencer quelque chose
\end{checkboxes}
%%%%%%%%%%%%%%%%%%%%%%%%%%%%%%%%%%%%%%%%%%%%%%%%%%%%%%%%%%%%%%%%%%%%%%%%%%%%%%%%

\question[1] Dans ce code, \mintinline{rust}{m} est passé comment?
\begin{checkboxes}
    \CorrectChoice Par deplacement \textit{move}
    \choice Par référence \textit{borrow} (mut/immutable)
    \choice Par copie \textit{copy}
\end{checkboxes}
\begin{minted}{rust}
struct Point { x: i32, y:i32 }
fn mystere(m: Point) -> Point {
    let mut m = m;
    m.x += 1;
    m.y += 42;
    m
}
\end{minted}

%%%%%%%%%%%%%%%%%%%%%%%%%%%%%%%%%%%%%%%%%%%%%%%%%%%%%%%%%%%%%%%%%%%%%%%%%%%%%%%%

\question[1] Anatomie d'un code Rust : associez les termes suivants au code suivant:
\begin{multicols}{2}
\begin{itemize}
    \item Opérateur d'addition
    \item Nom de variable
    \item Mot clef de déclaration de variable
    \item Argument de fonction
    \item Type
    \item Mot clef de déclaration de fonction
    \item Opérateur de d'enchaînement d'instruction
    \item Bloc du corps de la fonction
    \item Nom de fonction
    \item Appel de fonction associée à un type
    \item Argument de fonction passé en appel
    \item Expression du if
\end{itemize}
\end{multicols}
\begin{minted}{rust}
fn inconnue(a: &[i32], i: usize) -> Option<i32> {
    if a.len() < i {
        Some(a[i] + 50)
    } else {
        None
    }
}
\end{minted}
\vspace{2in}

%%%%%%%%%%%%%%%%%%%%%%%%%%%%%%%%%%%%%%%%%%%%%%%%%%%%%%%%%%%%%%%%%%%%%%%%%%%%%%%%
\question[1] Que signifie \mintinline{rust}{T} dans la signature de la fonction \mintinline{rust}{fn mystere<T>(a: T)}.
\vspace{1in}

%%%%%%%%%%%%%%%%%%%%%%%%%%%%%%%%%%%%%%%%%%%%%%%%%%%%%%%%%%%%%%%%%%%%%%%%%%%%%%%%
\question[1] Ce code peut-il compiler? Justifiez.
\begin{minted}{rust}
fn mystere(a: &mut [i32], b: &i32) {
    a[0] += *b;
}

fn main() {
    let mut a = [1, 2];
    mystere(&mut a, &a[0]);
}
\end{minted}
\vspace{1in}




%%%%%%%%%%%%%%%%%%%%%%%%%%%%%%%%%%%%%%%%%%%%%%%%%%%%%%%%%%%%%%%%%%%%%%%%%%%%%%%%
\question[1] Expliquez ce qu'est un \mintinline{rust}{trait} en Rust.
\vspace{1in}

%%%%%%%%%%%%%%%%%%%%%%%%%%%%%%%%%%%%%%%%%%%%%%%%%%%%%%%%%%%%%%%%%%%%%%%%%%%%%%%%

\question[1] A quoi sert le type \mintinline{rust}{Option<T>} et le type \mintinline{rust}{Result<T, E>}
\vspace{1in}



%%%%%%%%%%%%%%%%%%%%%%%%%%%%%%%%%%%%%%%%%%%%%%%%%%%%%%%%%%%%%%%%%%%%%%%%%%%%%%%%
\question[1] Proposez votre implementation de \mintinline{rust}{MyOption<T>} qui fait comme \mintinline{rust}{Option},
Et implementer la fonction \mintinline{rust}{MyOption::map} qui dois faire comme la documentation de \mintinline{rust}{Option::map}
de la lib standard.
\vspace{3in}

%%%%%%%%%%%%%%%%%%%%%%%%%%%%%%%%%%%%%%%%%%%%%%%%%%%%%%%%%%%%%%%%%%%%%%%%%%%%%%%%
\question[1] \mintinline{rust}{u32} est il passé par copie ou par move par défaut. 
\vspace{1in}

%%%%%%%%%%%%%%%%%%%%%%%%%%%%%%%%%%%%%%%%%%%%%%%%%%%%%%%%%%%%%%%%%%%%%%%%%%%%%%%%
\question[1] Ce code comporte une erreur, laquelle justifiez. Que fait \mintinline{rust}{?}.
\begin{minted}{rust}
fn mystere(a: Option<u32>) -> Option<u32> {
    let a = a?;
    Some(a + 1);
}
\end{minted}
\vspace{1.5in}

%%%%%%%%%%%%%%%%%%%%%%%%%%%%%%%%%%%%%%%%%%%%%%%%%%%%%%%%%%%%%%%%%%%%%%%%%%%%%%%%
\question[1] Pour un point realisez une implementation du trait \mintinline{rust}{Display} et \mintinline{rust}{Add} entre deux \mintinline{rust}{Point}.
\begin{minted}{rust}
struct Point {
    x: f32,
    y: f32,
}















\end{minted}
\end{questions}
\end{document}
