\documentclass[11pt,a4paper,addpoint]{exam}
\usepackage[T1]{fontenc}
\usepackage[utf8]{inputenc}
\usepackage[]{lmodern}
\usepackage[french]{babel}
%\usepackage{mathtools}
\usepackage[margin=2cm]{geometry} %layout
\usepackage{graphicx}
\usepackage{booktabs} % for much better looking tables
% Put the bibliography in the ToC
\usepackage[nottoc,notlof,notlot]{tocbibind}
% Alter the style of the Table of Contents
\usepackage[titles]{tocloft}
\usepackage{multicol}

\usepackage{minted}
\usepackage{titling}
\setlength{\droptitle}{-3cm}

\nopointsinmargin
\pointformat{}

\usepackage[pdfauthor={Axel Viala},
  pdftitle={Seconde-session 2021-2022: L2 Programmation avancée},
  pagebackref=true,%
  colorlinks=true,%
  linkcolor=green,%
  %urlcolor=green!70!black,
  pdftex]{hyperref}
\usepackage[ampersand]{easylist}
\renewcommand{\solutiontitle}{}

\author{\normalsize{Axel Viala <axel.viala@darnuria.eu>}}
\title{\normalsize{\textbf{Seconde-session 2021-2022: Programmation avancée}}}

\begin{document}
  \maketitle
  \begin{multicols}{2}

  \makebox[\textwidth][l]{Nom et Prénom:}
  \makebox[\textwidth][l]{Numéro étudiant:}
  \end{multicols}
  \textbf{Objectifs:} La clarté des réponses sera appréciée, veillez à écrire soigneusement. Les questions portent sur le langage Rust.
  Notes de cours autorisé. Réponse sur une copie a part encouragée.
  \newline
  \begin{questions}

    \section{Généralités}

%%%%%%%%%%%%%%%%%%%%%%%%%%%%%%%%%%%%%%%%%%%%%%%%%%%%%%%%%%%%%%%%%%%%%%%%%%%%%%%%

    \question[1] Par défaut les déclarations de variables sont:
    \begin{checkboxes}
        \CorrectChoice immutables
        \choice mutables
    \end{checkboxes}

\question[1] Toute valeur par exemple \mintinline{rust}{struct A} hors type de base est:
 \begin{checkboxes}
    \choice Pris par réfèrence
    \CorrectChoice Déplacé
    \choice Copié
\end{checkboxes}
%%%%%%%%%%%%%%%%%%%%%%%%%%%%%%%%%%%%%%%%%%%%%%%%%%%%%%%%%%%%%%%%%%%%%%%%%%%%%%%%
\question[1] Decrivez brievement les différentes formes de passage pour les valeurs:
\begin{itemize}
    \item Par deplacement \textit{move}
    \item Par référence \textit{borrow} (mut/immutable)
    \item Par copie \textit{copy}
\end{itemize}
\vspace{1in}


%%%%%%%%%%%%%%%%%%%%%%%%%%%%%%%%%%%%%%%%%%%%%%%%%%%%%%%%%%%%%%%%%%%%%%%%%%%%%%%%
\question[1] Anatomie d'un code Rust : associez les termes suivants au code suivant:
\begin{multicols}{2}
\begin{itemize}
    \item Opérateur d'addition
    \item Nom de variable
    \item Mot clef de déclaration de variable
    \item Argument de fonction
    \item Type
    \item Mot clef de déclaration de fonction
    \item Opérateur de d'enchaînement d'instruction
    \item Bloc du corps de la fonction
    \item Nom de fonction
    \item Appel de fonction associée à un type
    \item Argument de fonction passé en appel
\end{itemize}
\end{multicols}
\begin{minted}{rust}
fn foo(a: i32, b: i32) -> i32 {
    let t = a.min(b);
    t + a + b
}
\end{minted}
\vspace{2in}

\question[1] Donnez une signature de fonction polymorphique/générique en Rust, expliquez succinctement l'intéret du polymorphisme.
\vspace{1.5in}

%%%%%%%%%%%%%%%%%%%%%%%%%%%%%%%%%%%%%%%%%%%%%%%%%%%%%%%%%%%%%%%%%%%%%%%%%%%%%%%%
\question[1] Affichage et return, dans le code ci dessous, qu'afficherait le
programme s'il appellait \mintinline{rust}{mystere(1)} une fois? Quelle est la valeur de retour de mystere avec cet appel?
\begin{minted}{rust}
fn mystere(a: i32) -> i32 {
    println!("Mon mystère: {}", a);
    a + 42
}
\end{minted}
\vspace{1in}

%%%%%%%%%%%%%%%%%%%%%%%%%%%%%%%%%%%%%%%%%%%%%%%%%%%%%%%%%%%%%%%%%%%%%%%%%%%%%%%%
\question[1] Dans ce code, comment est passé \mintinline{rust}{a}, comment est passé \mintinline{rust}{b}.
Ce code compile t'il?
\begin{minted}{rust}
fn surprise(a: &mut i32, b: i32) {
    *a += b;
}
\end{minted}
\vspace{1in}

%%%%%%%%%%%%%%%%%%%%%%%%%%%%%%%%%%%%%%%%%%%%%%%%%%%%%%%%%%%%%%%%%%%%%%%%%%%%%%%%
\question[1] La fonction suivante peut-elle compiler ? Justifiez votre réponse.
\begin{minted}{rust}
fn mystere(a: i32) {
    match a {
        0 => 1,
        1 => 2 * a,
        n => a - a * 2,
    }
}
\end{minted}
\vspace{1in}
\pagebreak

\question[1] Rédigez le code néccessaire pour que \texttt{add\_one} ait le comportement attendu dans sa documentation. Plusieurs solutions possibles.
\texttt{unwrap}, \texttt{panic}, \texttt{unsafe} et \texttt{except} interdit.
\begin{minted}{rust}
/// Documentation de la fonction add_one
/// Ajoute la valeur derriere x, a un nombre contenu dans un Option, sinon None.
/// This function n'appelle pas `panic!()`.
fn add_one(a: Option<i32>, x: &i32) -> Option<i32> {





}
\end{minted}

%%%%%%%%%%%%%%%%%%%%%%%%%%%%%%%%%%%%%%%%%%%%%%%%%%%%%%%%%%%%%%%%%%%%%%%%%%%%%%%%
\question[1] Implémentations de fonctions sur un type. On souhaite calculer
la distance entre deux points via un \texttt{trait} .

\begin{minted}{rust}
struct Point {
    x: f32,
    y: f32,
}

trait Distance {
    fn distance(&self, other: &Self) -> f32;
}

// A vous de completer les _____

____ ___________  for __________ {

    // Completer les trou et ____.
    /// Calcule la distance entre deux `Points' via la formule
    /// `distance = sqrt((x - x') * (x - x'), (y - y') * (y - y'))`.
    /// Note: sqrt existe dans le module f32 et se nomme: sqrt()`
    fn ______(                             ) -> f32 {
        // A completer






    }
}
\end{minted}
%%%%%%%%%%%%%%%%%%%%%%%%%%%%%%%%%%%%%%%%%%%%%%%%%%%%%%%%%%%%%%%%%%%%%%%%%%%%%%%%

\question[1] Écrire une fonction \mintinline{rust}{simple_open} pour ouvrir un fichier sinon renvoyer une erreur, de signature:\\
\mintinline{rust}{fn simple_open(path: &str) -> Result<File, SimpleError>}


On dispose de \mintinline{rust}{pub fn open<P: AsRef<Path>>(path: P) -> Result<File>}.
\mintinline{rust}{path} qui peut être une chaine de caractères.

On souhaite en cas d'erreur une Erreur generique \mintinline{Rust}{Result::Err(SimpleError)}  ou renvoyer \mintinline{rust}{Ok(File)}. indice: \mintinline{rust}{match} ou \mintinline{rust}{map_err}.
\mintinline{rust}{SimpleError} sera une unit struct.
\vspace{1in}

\section{Implémentation d'une machine virtuelle}

Dans cette section il est proposé de réaliser une machine virtuelle a pile, aura une mémoire d'instructions representée par un \texttt{Vec<Instruction>} et une pile de travail representée par un \texttt{Vec<i16>}.

On ne \textbf{réalisera pas} la transformation du texte vers notre evaluateur. Pour cet exercice tout sera fondé sur votre énumeration \texttt{Instruction}.

La machine aura les instructions suivantes :

\begin{itemize}
    \item Agissant sur les deux entiers au sommet de pile: Addition, Soustraction
    \item Pousse: Ajoute un \texttt{i16} sur la pile
    \item Duplique: Duplique le sommet de pile
    \item Supprime: Retire le sommet de pile
\end{itemize}

Votre machine virtuelle et son implémentation pourrons executer: \texttt{1 1 + duplique +} ce qui donne en décomposé : \texttt{1 1 +} soit \texttt{2} sur la pile, puis duplique \texttt{2} donc \texttt{2 2} en pile puis \texttt{+} soit \texttt{4} sur le sommet de pile.

\mintinline{rust}{ExecError::PileInsuffisante} est retourné en cas d'absence de suffisament d'élements sur la pile pour une instruction.
La gestion des erreurs est valorisé a la notation.

\question[1] Réaliser une enumération \texttt{Instruction} representant les instructions décrite ci dessus.

\question[1] Écrire une structure \texttt{VirtualMachine} qui contient notre tableau d'\texttt{Instructions}, la pile.
Indice: utilisez des \texttt{Vec}, et tout champ que vous jugerez pertinent pour votre implémentation.

\question[1] Écrire une enumération \texttt{ExecError} qui contient un unique variant \texttt{PileInsuffisante}.

\question[1] Écrire les fonctions \texttt{new}, \texttt{eval} associées à la structure \texttt{VirtualMachine}.
\begin{itemize}
    \item \texttt{new} permet de construire une machine virtuelle avec une pile d'instructions passée par déplacement. La pile de travail sera initialisée vide.
    \item \texttt{eval} fonction qui execute les instructions dans la pile d'instructions. Cette fonction renvoie un \texttt{Result<(), ExecError>}.
\end{itemize}
\vspace{5in}


\end{questions}
\end{document}
