\documentclass[11pt,a4paper,addpoint]{exam}
\usepackage[T1]{fontenc}
\usepackage[utf8]{inputenc}
\usepackage[]{lmodern}
\usepackage[french]{babel}
%\usepackage{mathtools}
\usepackage[top=1 cm, bottom=1 cm, left=1.5 cm, right=1.5 cm]{geometry} %layout
\usepackage{graphicx}
\usepackage{booktabs} % for much better looking tables
% Put the bibliography in the ToC
\usepackage[nottoc,notlof,notlot]{tocbibind}
% Alter the style of the Table of Contents
\usepackage[titles]{tocloft}

\usepackage[pdfauthor={Axel Viala},
  pdftitle={QCM 01 L2 Paris 8 Programmation avancée Rust},
  pagebackref=true,%
  colorlinks=true,%
  linkcolor=green,%
  %urlcolor=green!70!black,
  pdftex]{hyperref}
\usepackage[ampersand]{easylist}
\renewcommand{\solutiontitle}{}

\usepackage{minted}
\nopointsinmargin
\pointformat{}

\author{\small{Axel Viala <axel@darnuria.eu>}}
\title{\small{CC01: Questions de code mystère}}
\date{\small{\today}}

\begin{document}
  \maketitle
  \makebox[\textwidth][l]{NOM et Prénom:\hrulefill}
  \makebox[\textwidth][l]{Groupe : L2 - A}
    
%   \textbf{Rendu:} Vous devez rendre ce devoir, avant le dimanche 31 janvier 18h par courriel avec le sujet
%   \texttt{«[rust-esgi] QCM "votre promotion" "nom" "prenom"»} a mon adresse \emph{axel.viala@darnuria.eu}.
%   Vous pouvez soit répondre directement sur le PDF, soit imprimer et scanner/(scanner avec ordiphone) et m'envoyer le scan.
%   Le QCM est à faire seul, la clarté sera un plus, les mauvaises réponses font perdre des points.
%   \textbf{Objectifs:} Le but du contrôle de connaissances en début de cours est pour vous de vérifier où vous
%   en êtes par rapport au cours précédent.
%   \newline
%   Il s'agit pour moi un moyen de vérifier que la pédagogie est adaptée à la classe.
%   \textbf{Notation:} Les points sont indiqués à titre d'information, la notation peut changer pour des raisons d'harmonisation. Les réponses fausses en QCM font perdre des points.  
  \begin{questions}

\section{Rust syntaxe, typage, concepts de base}

%%%%%%%%%%%%%%%%%%%%%%%%%%%%%%%%%%%%%%%%%%%%%%%%%%%%%%%%%%%%%%%%%%%%%%%%%%%%%%%%
\question[1] Ce code peut-il compiler? Pourquoi dans les deux cas:
\begin{minted}{rust}
let mut a = vec![1, 2, 4, 5, 12];
let mystère = &a[4];
a.pop();
println!("{}", mystère);
\end{minted}
\begin{itemize}
    \item \mintinline{rust}{&a[i]} corresponds à \mintinline{Rust}{fn index(Vec<T>, index: usize) -> &T} 
    \item \mintinline{rust}{pub fn pop(&mut self) -> Option<T>}
    \item \mintinline{rust}{T} est un \mintinline{rust}{i32}
\end{itemize}

\ifprintanswers
\begin{solution}
    Ne compile pas: On ne peux pas emprunter \mintinline{rust}{&a[4]} et ensuite \mintinline{rust}{pop} c'est une invalidation de référence Rust ne laisse pas compiler par les règles du borrowing.
\end{solution}
\else
\vspace{3in}
\fi

%%%%%%%%%%%%%%%%%%%%%%%%%%%%%%%%%%%%%%%%%%%%%%%%%%%%%%%%%%%%%%%%%%%%%%%%%%%%%%%%
\question[1] Quel type represente une valeur qui peut-être absente en Rust. Donner une définition de ce type en Rust et un exemple d'usage.

\ifprintanswers
\begin{solution}

\end{solution}
\else
\vspace{1in}
\fi

\newpage
%%%%%%%%%%%%%%%%%%%%%%%%%%%%%%%%%%%%%%%%%%%%%%%%%%%%%%%%%%%%%%%%%%%%%%%%%%%%%%%%
\question[1] Soit le code suivant:
\begin{minted}{rust}
let a = vec![1, 42, 4, 1];
let b = a;
println!("{:?}", a);
\end{minted}
Est-ce que ce code compile? Pourquoi dans les deux cas?
\ifprintanswers
\begin{solution}

\end{solution}
\else
\vspace{2in}
\fi

%%%%%%%%%%%%%%%%%%%%%%%%%%%%%%%%%%%%%%%%%%%%%%%%%%%%%%%%%%%%%%%%%%%%%%%%%%%%%%%%
\question[1] À quoi sert le mot clef \mintinline{rust}{match}? À quoi sert une \mintinline{rust}{enum}? Donner le cas obligatoirement pour une \mintinline{rust}{enum} et eventuellemnent les autres cas.
\ifprintanswers
\begin{solution}

\end{solution}
\else
\vspace{3in}
\fi

%%%%%%%%%%%%%%%%%%%%%%%%%%%%%%%%%%%%%%%%%%%%%%%%%%%%%%%%%%%%%%%%%%%%%%%%%%%%%%%%
\question[1] Expliquer ce qu'est l'immutabilité, les références et les règles importantes de partage associées et le concept d'ownership (propriété). Bonus: Tentez d'expliquer quels bugs sont exclus grace à ces régles et dans quel cas sur des procésseurs modernes cela peut-être pratique.
\ifprintanswers
\begin{solution}

\end{solution}
\else
\vspace{1in}
\fi

\end{questions}
\end{document}
