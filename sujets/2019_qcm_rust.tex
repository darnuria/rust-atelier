\documentclass[11pt,a4paper,addpoint,answers]{exam}
\usepackage[T1]{fontenc}
\usepackage[utf8]{inputenc}
\usepackage[]{lmodern}
\usepackage[french]{babel}
%\usepackage{mathtools}
\usepackage[margin=2cm]{geometry} %layout
\usepackage{graphicx}
\usepackage{booktabs} % for much better looking tables
% Put the bibliography in the ToC
\usepackage[nottoc,notlof,notlot]{tocbibind}
% Alter the style of the Table of Contents
\usepackage[titles]{tocloft}

\usepackage[pdfauthor={Axel Viala},
  pdftitle={QCM ESGI Rust},
  pagebackref=true,%
  colorlinks=true,%
  linkcolor=green,%
  %urlcolor=green!70!black,
  pdftex]{hyperref}
\usepackage[ampersand]{easylist}
\renewcommand{\solutiontitle}{}

\usepackage{minted}
\nopointsinmargin
\pointformat{}

\author{Axel Viala <axel.viala@darnuria.eu>}
\title{Devoir maison: Programmation systèmes et Réseaux en Rust, contrôle de connaissances}
\date{21 janvier 2019}

\begin{document}
  \maketitle
  \makebox[\textwidth][l]{Nom et Prénom:\hrulefill}
  \makebox[\textwidth][l]{Classe:\hrulefill}

  \textbf{Rendu:} Vous devez rendre ce devoir, avant le dimanche 31 janvier 18h par courriel avec le sujet
  \texttt{«[rust-esgi] QCM "votre promotion" "nom" "prenom"»} a mon adresse \emph{axel.viala@darnuria.eu}.
  Vous pouvez soit répondre directement sur le PDF, soit imprimer et scanner/(scanner avec ordiphone) et m'envoyer le scan.
  Le QCM est à faire seul, la clarté sera un plus, les mauvaises réponses font perdre des points.
  \textbf{Objectifs:} Le but du contrôle de connaissances en début de cours est pour vous de vérifier où vous
  en êtes par rapport au cours précédent.
  \newline
  Il s'agit pour moi un moyen de vérifier que la pédagogie est adaptée à la classe.
  \newline
  \textbf{Notation:} Les points sont indiqués à titre d'information, la notation peut changer pour
  des raisons d'harmonisation. Les réponses fausses en QCM font perdre des points.
  \begin{questions}

    \section{Culture générale autour de Rust}

    \question[1] En Rust, avez-vous à écrire les types par vous-même:
    \begin{checkboxes}
        \choice Oui comme en C++ sans \mintinline{C++}{auto}
        \CorrectChoice Non le compilateur infére les types et parfois me demande du secours
        \choice Comme en JavaScript ou Python les types sont gérées à l'exécution
    \end{checkboxes}

    \question[1] En Rust, quelles propositions sont vraies:
    \begin{checkboxes}
        \choice Rust gère la mémoire avec un gabarge collector
        \CorrectChoice En Rust les allocations sont gérées par scope, le compilateur place les appels aux destructeurs
        \choice En Rust par défaut \texttt{rustc} compile pour une machine virtuelle
        \CorrectChoice \texttt{rustc} compile en assembleur natif pour mon ordinateur
        \choice La mémoire allouée est dans le segment de tas (HEAP) sans que je le demande explicitement
        \choice C'est moi qui appelle \mintinline{C}{free} pour libérer les allocations sur le tas/heap
        \choice La gestion par défaut de la mémoire est gérée par un compteur de référence comme en Swift
    \end{checkboxes}
\newpage
    \section{Syntaxe de Rust}

    \question[1] En Rust, le mot clef \mintinline{rust}{let} sert à:
    \begin{checkboxes}
        \choice Faire un branchement conditionnel \emph{Il existe le \mintinline{rust}{if-let} mais c'est pas vraiment un branchement classique.}
        \CorrectChoice Lier une expression à un nom (déclarer une variable).
        \choice Par défaut la variable sera dans le tas/heap
        \CorrectChoice Par défaut la variable sera allouée dans la pile/STACK
        \CorrectChoice Il peut servir à destructurer une expression ex: \mintinline{rust}{let (a, b) = (1, 42);}
    \end{checkboxes}

    \question[1] En Rust, par défaut les variables sont:
    \begin{checkboxes}
        \CorrectChoice \emph{imutables} / constantes
        \choice mutables
    \end{checkboxes}

    \question[1] Que dois-je faire si je souhaite échanger entre plusieurs threads en Rust?
    \begin{checkboxes}
        \CorrectChoice Par mémoire partagée: avec un compteur atomique de références \mintinline{rust}{std::sync::Arc} \href{https://doc.rust-lang.org/std/sync/struct.Arc.html}{doc} et une \mintinline{rust}{std::sync::Mutex} \href{https://doc.rust-lang.org/std/sync/struct.Mutex.html}{doc}
        \CorrectChoice Par passage de messages: avec des channels: voir \mintinline{rust}{std::sync::mpsc} \href{https://doc.rust-lang.org/rust-by-example/std_misc/channels.html}{doc}
        \choice Une simple référence mutable suffit
        \choice Je ne peux pas
    \end{checkboxes}

    \question[1] A quoi sert \mintinline{rust}{std::sync::Mutex} ?
    \begin{checkboxes}
        \choice Compter des références
        \CorrectChoice Protéger un accès concurent à une ressource
        \choice Déterminer quand libérer la mémoire
    \end{checkboxes}

    \question[1] A quoi sert \mintinline{rust}{std::sync::Arc} ?
    \begin{checkboxes}
        \CorrectChoice Compter des références de façon atomique pour liberer si le compteur est a 0
        \choice Protéger un accès concurent à une ressource
        \CorrectChoice Déterminer quand libérer la mémoire (J'ai été généreux et pas compter l'oubli de cette réponse).
    \end{checkboxes}

    \question[1] Quelles propositions pour le code Rust suivant sont vraies.
\begin{minted}{rust}
// Rappel: Vec n'implemente pas copy.
let v = vec![2, 42, 1];
let s = v;
println!("{}", v[0]); // v[0] fait un borrow.
\end{minted}

\begin{checkboxes}
    \choice Ce code compile
    \CorrectChoice \mintinline{rust}{s} est un \emph{move} (capture), de \mintinline{rust}{v} donc \mintinline{rust}{v} n'est plus utilisable
    \choice \mintinline{rust}{s} est un \emph{borrow} (emprunt), de \mintinline{rust}{v}.
\end{checkboxes}

\question[1] Expliquer avec vos mots le concept de \emph{move semantics} (capture ou déplacement en Français)/.
\ifprintanswers
\begin{solution}
Un \emph{move} ou \emph{déplacement} est l'action lorsque une valeur change de propriétaire (\emph{owner}),
par exemple lors d'un appel de fonction, une liaison avec \mintinline{rust}{let} ou lors d'une capture dans une closure.

Cette variable n'est alors plus utilisable ayant été déplacée. Dans le cas ou on désire pas
de déplacement ni de copie on peut faire un emprunt.
Certains types implementent \emph{Copy et Clone} et donc le problème ne se pose pas (ex: \mintinline{rust}{i32}).
\end{solution}
\else
\vspace{2in}
\fi

\newpage
\question[1] En Rust, implémenter le \mintinline{rust}{trait Copy} fait que votre type sera:
    \begin{checkboxes}
        \CorrectChoice Dupliqué et passé par copie bit à bit.
        \CorrectChoice Après un passage de paramètre j'ai donc 2 occurences de mon types dans la mémoire
        \choice Modifier la copie modifie l'original
        \choice C'est gratuit en terme de performance pour les grosses structures
   \end{checkboxes}

    \question[1] Le type \mintinline{rust}{std::boxed::Box} sert a quoi? Dans quels cas en avez vous absolument
    besoin?
\ifprintanswers
\begin{solution}
\mintinline{rust}{std::boxed::Box} permet de faire des allocations sur le tas \emph{Heap} et manage sa
desallocation grace à l'ownership. Il s'agit d'un pointeur dit intelligent comparable à un \mintinline{c++}{std::unique_ptr<T>} de \emph{C++}.
\end{solution}
\else
\vspace{2in}
\fi

\question[1] A quoi sert le mot clef \mintinline{rust}{match}?
\begin{checkboxes}
    \CorrectChoice Faire du filtrage par motif \emph{pattern matching}
    \choice Faire des allocations sur la Heap
    \CorrectChoice Déconstruire des \mintinline{rust}{enum} selon leur contenu
    \choice Sur une \mintinline{rust}{enum} je peux filtrer seulement certains variants d'une \mintinline{rust}{enum}
    \CorrectChoice Sur une \mintinline{rust}{enum} je dois être exhaustif et filtrer tout les variants
    \CorrectChoice Je peux matcher autre chose que des \mintinline{rust}{enum} par exemle entiers et chaines
\end{checkboxes}

\question[1] En Rust je peux emprunter \emph{borrow} une référence sur une valeur avec \mintinline{rust}{&}
et \emph{borrow} mutablement avec \mintinline{rust}{mut &}. Quelles sont les règles?
\begin{checkboxes}
    \CorrectChoice Une valeur immutable peut être partagée imutablement plusieurs fois en lecture
    \CorrectChoice Une valeur mutable peut être partagée mutablement par un lecteur/écrivain
    \choice Une valeur mutable peut être partagée mutablement par plusieurs lecteur/écrivain
    \choice Une valeur immutable peut être partagée mutablement quand je veux
\end{checkboxes}

Soit le code suivant: \href{https://play.rust-lang.org/?version=stable&mode=debug&edition=2018&code=fn%20main()%20%7B%0A%20%20%20%20%2F%2F%20Indice%20les%20%26str%20n%27implementent%20pas%20%60Copy%60%2C%20ni%20%60Clone%60.%0A%20%20%20%20%2F%2F%20Indice2%3A%20Si%20vous%20%C3%AAtes%20pas%20Copy%20vous%20%C3%AAtes%20Move!%20%3B)%0A%20%20%20%20let%20mut%20s%20%3D%20vec!%5B42%2C%2021%2C%200%5D%3B%0A%20%20%20%20let%20a%20%3D%20%26s%5B1%5D%3B%0A%20%20%20%20s.remove(1)%3B%0A%20%20%20%20println!(%22%7B%7D%22%2C%20a)%3B%0A%7D}{code sur le playpen}

\begin{minted}{rust}
fn main() {
    // Indice: les &str n'implémentent pas `Copy`, ni `Clone`.
    // Indice2: Si vous n'êtes pas Copy vous êtes Move! ;)
    let mut s = vec![42, 21, 0];
    let a = &s[1];
    s.remove(1);
    println!("{}", a);
}
\end{minted}

\question[1] Expliquer succinctement pourquoi ce code ne compile t-il pas? (indice borrow mutable/immutable)
\ifprintanswers
\begin{solution}
Ici \emph{Rust} interdit
\end{solution}
\else
\vspace{2in}
\fi

\subsection{struct et question de mémoire}
\question[1] Soit le code suivant: (\href{https://play.rust-lang.org/?version=stable&mode=debug&edition=2018&code=use%20std%3A%3Aboxed%3A%3ABox%3B%0A%0Astruct%20Point%20%7B%0A%20%20%20%20x%3A%20i32%2C%0A%20%20%20%20y%3A%20i32%20%0A%7D%0A%0Afn%20main()%20%7B%0A%20%20%20%20let%20p%20%3D%20Point%20%7B%20x%3A%202%2C%20y%3A%204%7D%3B%0A%20%20%20%20let%20h%20%3D%20Box%3A%3Anew(Point%20%7B%20x%3A%206%2C%20y%3A%2042%20%7D)%3B%0A%7D}{lien playpen})
\begin{minted}{rust}
use std::boxed::Box;

struct Point {
    x: i32,
    y: i32
}

fn main() {
    let p = Point { x: 2, y: 4 };
    let h = Box::new(Point { x: 6, y: 42 });
}
\end{minted}

\question[1] Quelle place occupe la structure \mintinline{rust}{Point} en mémoire ? Indice: \href{https://doc.rust-lang.org/std/mem/fn.size_of.html}{doc size\_of}
    \begin{checkboxes}
        \CorrectChoice 64bits
        \choice 128bits
        \choice 42bits
        \choice 32bits
   \end{checkboxes}

\question[1] Où est située la valeur de \mintinline{rust}{p} ?
    \begin{checkboxes}
        \CorrectChoice Stack/Pile
        \choice Heap/Tas
   \end{checkboxes}

   \question[1] Où est située la valeur de \mintinline{rust}{h} ?
    \begin{checkboxes}
        \choice Stack/Pile
        \CorrectChoice Heap/Tas
   \end{checkboxes}

   \question[1] Où est située le pointeur \mintinline{rust}{h} ?
   \begin{checkboxes}
       \CorrectChoice Stack/Pile
       \choice Heap/Tas
  \end{checkboxes}

  \question[1] Dessiner comment est organisée la mémoire pile et tas à la fin de la fonction \mintinline{rust}{main()} avant les dessallocations.
  Préciser ce qui est le Tas/Heap, la Pile/Stack et travers les éventuels pointeurs. Indice: les questions précédentes devraient être des
  indices.
  \ifprintanswers
\begin{solution}
\end{solution}
\else
\vspace{2in}
\fi


\question[1] Le type \mintinline{rust}{std::boxed::Box} sert a quoi? Dans quels cas en avez vous absolument
besoin?
\ifprintanswers
\begin{solution}
\end{solution}
\else
\vspace{2in}
\fi

\subsection{Enum, mémoire et developpement}

Soit un type \mintinline{rust}{Peano} qui implémente des nombres de \emph{Peano}.
\href{https://play.rust-lang.org/?version=stable&mode=debug&edition=2018&code=use%20std%3A%3Aboxed%3A%3ABox%3B%0A%0Aenum%20Peano%20%7B%0A%20%20%20%20Zero%2C%0A%20%20%20%20S(Box%3CPeano%3E)%0A%7D%0A%0Afn%20main()%20%7B%0A%20%20%20%20let%20z%20%3D%20Peano%3A%3AZero%3B%0A%20%20%20%20let%20h%20%3D%20Peano%3A%3AS(Box%3A%3Anew(Peano%3A%3AZero))%3B%0A%7D}{lien playppen}
\begin{minted}{rust}
use std::boxed::Box;

enum Peano {
    Zero,
    S(Box<Peano>)
}

fn main() {
    let z = Peano::Zero;
    let h = Peano::S(Box::new(Peano::Zero));
}
\end{minted}

\question[1] Où est située la valeur de \mintinline{rust}{z} ?
    \begin{checkboxes}
        \CorrectChoice Stack/Pile
        \choice Heap/Tas
   \end{checkboxes}

   \question[1] Où est située la valeur de \mintinline{rust}{h} ?
    \begin{checkboxes}
        \CorrectChoice La valeur du variant \mintinline{rust}{Peano::S} est dans la Stack/Pile
        \CorrectChoice \mintinline{rust}{Zero} dans le Heap/Tas
   \end{checkboxes}

\question[1] Dessiner comment est organisée la mémoire pile et tas à la fin de la fonction \mintinline{rust}{main()}
avant les dessallocations.
Préciser ce qui est le Tas/Heap, la Pile/Stack et travers les eventuels pointeurs. Indice: les questions précédentes devraient être des
indices.
\ifprintanswers
\begin{solution}
\end{solution}
\else
\vspace{2in}
\fi

On veut pouvoir écrire des fonctions pour manipuler nos nombres de \emph{Peano}.
Vous devrez dans les questions suivantes, remplacer les \emph{???}, écrire le code des fonctions dans un \textbf{fichier} nommée \texttt{nom\_prenom\_promo.rs}.

Indice: vous aurez besoin du mot clef \mintinline{rust}{match} et pourquoi pas de récursivité pour vous simplifier l'écritude du code.
\href{https://play.rust-lang.org/?version=stable&mode=debug&edition=2018&code=use%20std%3A%3Aboxed%3A%3ABox%3B%0A%0Aenum%20Peano%20%7B%0A%20%20%20%20Zero%2C%0A%20%20%20%20S(Box%3CPeano%3E)%0A%7D%0A%0Aimpl%20Peano%20%7B%0A%20%20%20%20%2F%2F%2F%20Cr%C3%A9e%20un%20nombre%20de%20peano%20depuis%20un%20nombre%20natif%20sur%2032bits.%0A%20%20%20%20%2F%2F%2F%20%60%60%60rust%0A%20%20%20%20%2F%2F%2F%20assert_eq(new_from_i32(2)%2C%20Peano%3A%3AS(Box%3A%3Anew(Peano%3A%3AS(Box%3A%3Anew(Peano%3A%3AZero))))%3B%0A%20%20%20%20%2F%2F%2F%20%60%60%60%0A%20%20%20%20fn%20new_from_i32(n%3A%20u32)%20-%3E%20Peano%20%7B%0A%20%20%20%20%20%20%20%20%0A%20%20%20%20%7D%0A%20%20%20%20%0A%20%20%20%20%2F%2F%2F%20Transforme%20un%20Peano%20en%20nombre%0A%20%20%20%20%2F%2F%2F%20%60%60%60rust%0A%20%20%20%20%2F%2F%2F%20assert_eq(Peano%3A%3AS(Box%3A%3Anew(Peano%3A%3AZero)).to_i32()%2C%201)%3B%0A%20%20%20%20%2F%2F%2F%20%60%60%60%0A%20%20%20%20fn%20to_i32(%3F%3F%3F)%20-%3E%20u32%20%7B%0A%20%20%20%20%20%20%20%20%0A%20%20%20%20%7D%0A%20%20%20%20%0A%20%20%20%20%2F%2F%2F%20Additionne%20deux%20Peano%20entre%20eux.%0A%20%20%20%20%2F%2F%2F%20Exemple%0A%20%20%20%20%2F%2F%2F%20%60%60%60rust%0A%20%20%20%20%2F%2F%2F%20assert_eq!(p%2C%20p.add(Peano%3A%3AZero))%3B%0A%20%20%20%20fn%20add(%3F%3F%3F%2C%20%3F%3F%3F)%20-%3E%20%3F%3F%3F%20%7B%0A%0A%20%20%20%20%7D%0A%7D%0A%0Afn%20main()%20%7B%0A%20%20%20%20let%20z%20%3D%20Peano%3A%3AZero%3B%0A%20%20%20%20let%20h%20%3D%20Peano%3A%3AS(Box%3A%3Anew(Peano%3A%3AZero))%3B%0A%7D}{lien playpen rust}
\begin{minted}{rust}
impl Peano {
    /// Crée un nombre de peano depuis un nombre natif sur 32bits.
    /// ```rust
    /// assert_eq(new_from_i32(2), Peano::S(Box::new(Peano::S(Box::new(Peano::Zero))));
    /// ```
    fn new_from_i32(n: u32) -> Peano {

    }

    /// Transforme un Peano en nombre
    /// ```rust
    /// assert_eq(Peano::S(Box::new(Peano::Zero)).to_i32(), 1);
    /// ```
    fn to_i32(???) -> u32 {

    }

    /// Additionne deux Peano entre eux.
    /// Exemple
    /// ```rust
    /// assert_eq!(p, p.add(Peano::Zero));
    fn add(???, ???) -> ??? {

    }
}
\end{minted}
\question[1] Écrivez le code de la fonction \mintinline{rust}{new_from_i32}.

\ifprintanswers
\begin{solution}
\end{solution}
\else
\vspace{2in}
\fi

\question[1] Écrivez le code de la fonction \mintinline{rust}{to_i32}.

\ifprintanswers
\begin{solution}
\end{solution}
\else
\vspace{2in}
\fi

\question[1] Écrivez le code de la fonction \mintinline{rust}{add}.

\ifprintanswers
\begin{solution}
\end{solution}
\else
\vspace{2in}
\fi

  \end{questions}
\end{document}
